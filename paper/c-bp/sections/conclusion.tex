% !TEX root = ../Asymptotic-CBP.tex

%%%%%%%%%%%%%%%%%%%%%%%%%%%%%%%%%%%%%%%%%%%%%%%%%%%
\section*{Conclusion}

In this work, we have provided a precise analysis of the properties of the solution path of $\ell^1$-type variational problems in the low-noise regime. This includes in particular the \lasso and the C-BP problems. A particular attention has been paid to the support set of this path, which in general cannot be expected to match the one of the sought after solution. Two striking examples support the relevance of this approach. For the deconvolution problem, we showed theoretically that in general this support is not stable, and we were able to derive in closed form the solution of the ``extended support'' that is twice larger, but is stable. In the compressed sensing scenario (i.e. when the operator of the inverse problem is random), we showed numerically how to leverage our theoretical findings and analyze the growth of the extended support size as the number of measurements diminishes. This analysis opens the doors for many new developments to better understand this extended support, both for deterministic operators (e.g. Radon transform in medical imaging) and random ones.  


%%%%%%%%%%%%%%%%%%%%%%%%%%%%%%%%%%%%%%%%%%%%%%%%%%%
\section*{Acknowledgements} 

We would like to thank Charles Dossal, Jalal Fadili and Samuel Vaiter for stimulating discussions on the notion of extended support. This work has been supported by the European Research Council (ERC project SIGMA-Vision).
