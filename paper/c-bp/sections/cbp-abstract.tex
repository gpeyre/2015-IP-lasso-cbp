% !TEX root = ../Asymptotic-CBP.tex

\section{Abstract analysis of the \protect\lasso with cone constraint}
\label{sec-continuous-abstract}

This section studies a simple variant of the \lasso with cone constraint in an abstract setting. The results stated here shall be useful in Section~\ref{sec-contbp-thin}, since this variant turns out to be the Continuous Basis-Pursuit when the degradation operator is the integration of an impulse response and its derivative.

%%%%%%%%%%%%%%%%%%%%%%%%%%%%%%%%%%%%%%%%%%%%%%%%%%%%%%
\subsection{Notations}

Given a parameter $\stepsize>0$, we consider the convex cone generated by the vectors $(1,\frac{\stepsize}{2})$ and $(1,-\frac{\stepsize}{2})$,
\eql{\label{eq-C-cone}
	\coneh \eqdef \enscond{(c,d) \in \RR \times \RR}{ c\geq 0 \qandq  -c\frac{\stepsize}{2}+|d|\leq 0 }.
 }
 %$e_\ell = (1,\ell) \in \RR^2$, for $\ell \in \{+\frac{\stepsize}{2},-\frac{\stepsize}{2}\}$.
We also define the cone $\coneh^\taillegrid$ as the set of vectors $(\veccont,\shiftcont)\in \RR^\taillegrid\times \RR^\taillegrid$ such that for all $k\in\seg{0}{\taillegrid-1}$ $(\veccont_k,\shiftcont_k)\in \coneh$.

Now, given a vector $(\veccontO,\shiftcontO)\in \coneh^\taillegrid$ (\ie  $\forall k\in \seg{0}{\taillegrid-1}$, $\veccont_{0,k}\geq \frac{2}{\stepsize}|\shiftcont_{0,k}|$), we observe $y_0\eqdef\OpU \veccontO + \OpD \shiftcontO$, where $\OpU: \RR^\taillegrid\rightarrow \Hh$ and $\OpD: \RR^\taillegrid\rightarrow \Hh$ are linear operators, or its noisy version $y=y_0+w$ where $w\in \Hh$.
To recover $(\veccontO,\shiftcontO)$ from $y$ or $y_0$, we consider the following reconstruction problems:
\begin{align}
  \umin{(\veccont,\shiftcont)\in \coneh^\taillegrid} \frac{1}{2}\normH{y-\OpU \vecdisc-\OpD\shiftcont}^2 + \la \norm{\veccont}_1, \tag{$\Qq_\la(y)$}\label{eq-abstract-cbpasso}
\end{align}
 and for $\la =0$,
\begin{align}
  \umin{(\veccont,\shiftcont)\in \coneh^\taillegrid} \norm{\veccont}_1 \mbox{ such that } \OpU \veccont+ \OpD \shiftcont=y_0.  \tag{$\Qq_0(y_0)$}\label{eq-abstract-cbp}
\end{align}

Our main focus is on the support recovery properties of~\eqref{eq-abstract-cbpasso}. Precisely, we split the ``support'' of $(\veccont,\shiftcont)\in\coneh^\taillegrid$ into several parts:
\begin{align}
  I~&\eqdef \supp(\veccont) \eqdef \enscond{i \in \seg{0}{\taillegrid-1}}{\veccont_i > 0}\\
&=\Iup\cup\Idown\\
\mbox{where } \Iup &\eqdef \enscond{i\in I}{\veccont_i +\frac{2}{\stepsize}\shiftcont_i >0 },\quad 
	\Idown \eqdef \enscond{i\in I}{\veccont_i -\frac{2}{\stepsize}\shiftcont_i >0 }.
\end{align}
In general $\Iup\cap \Idown\neq \emptyset$. If $(\veccont_\la, \shiftcont_\la)$ is a solution of~\eqref{eq-abstract-cbpasso}, we say that we have exact support recovery provided that $\Iup(\veccont_\la, \shiftcont_\la)= \Iup(\veccontO, \shiftcontO)$ and $\Idown(\veccont_\la, \shiftcont_\la)= \Idown(\veccontO, \shiftcontO)$. 

\begin{rem}
  The notation $\Iup$, $\Idown$ shall become clearer in the next section.
  When considering the Continuous Basis-Pursuit on a grid with stepsize $\stepsize>0$, points $i$ in $\Iup$ correspond to Dirac masses which ``tend to be on the right'', that is they do not coincide with the left half-grid point $i\stepsize-\frac{\stepsize}{2}$. Similarly, points in $\Idown$ correspond to Dirac masses which ``tend to be on the left'', as they do not coincide with the right half-grid point $i\stepsize+\frac{\stepsize}{2}$. In fact, if $i\in \Iup\setminus\Idown$, it correponds to a Dirac mass at the right half-grid point: $\delta_{i\stepsize+\frac{\stepsize}{2}}$, and if $i\in \Idown\setminus\Iup$, it correponds to a Dirac mass  at the left half-grid point: $\delta_{i\stepsize-\frac{\stepsize}{2}}$. If $i\in \Iup\cap \Idown$, it correponds to a Dirac mass which may belong ``freely'' to the interval $(i\stepsize-\frac{\stepsize}{2},i\stepsize+\frac{\stepsize}{2})$.  \todo{Figure}
\end{rem}


%%%%%%%%%%%%%%%%%%%%%%%%%%%%%%%%%%%%%%%%%%%%%%%%%%%%%%
\subsection{Parametrization as a positive \protect\lasso }
\label{sec-cbp-another-param}

To characterize the solutions of~\eqref{eq-abstract-cbpasso} and~\eqref{eq-abstract-cbp}, it is convenient to reparametrize the problem as a \lasso with positivity constraint, writing for all $i\in\seg{0}{\taillegrid-1}$,
\begin{align}
	\begin{pmatrix}
\veccont_i\\\shiftcont_i
\end{pmatrix}  \eqdef  \begin{pmatrix}
    1&1\\ \frac{\stepsize}{2}&-\frac{\stepsize}{2}
  \end{pmatrix} \begin{pmatrix}
    \parU_i\\ \parD_i
  \end{pmatrix}
 	\quad\text{or}\quad
 	\begin{pmatrix}
    \parU_i\\\parD_i
  \end{pmatrix}=\frac{1}{2} \begin{pmatrix}
    \veccont_i+\frac{2}{\stepsize} \shiftcont_i\\
    \veccont_i-\frac{2}{\stepsize} \shiftcont_i\\
  \end{pmatrix}. 
\label{eq-cbp-cdv}
 \end{align}
In the following, we define the linear map
\eql{\label{eq-def-H_h}
	H_\stepsize: 
	\begin{pmatrix} \parU \\ \parD \end{pmatrix}
	\longmapsto
	\begin{pmatrix} \veccont \\ \shiftcont \end{pmatrix}
} 
 
It is clear that $(\veccont_i, \shiftcont_i)\in\coneh$ if and only if $\parU_i\geq 0$ and $\parD_i\geq 0$. Moreover, given $(\veccont, \shiftcont)\in \coneh^\taillegrid$,
  \begin{align*}
    I^c&=\enscond{i\in\seg{0}{\taillegrid-1}}{(\parU_i,\parD_i)=(0,0)},\\
    \Iup &=\enscond{i\in\seg{0}{\taillegrid-1}}{\parU_i> 0}, 
\qandq    \Idown=\enscond{i\in\seg{0}{\taillegrid-1}}{\parD_i>0}.
  \end{align*}

%More globally, we shall write this transformation $(\veccont,\shiftcont)=H\begin{pmatrix}
%\parU\\ \parD
%\end{pmatrix}$ where $H\in \RR^{2\taillegrid\times 2\taillegrid }$ and  $(\parU,\parD)\in\RR^{\taillegrid}\times \RR^{\taillegrid}$.

Therefore, Problems~\eqref{eq-abstract-cbpasso} and~\eqref{eq-abstract-cbp} are respectively equivalent to the \lasso and Basis Pursuit with positivity constraint:
\begin{align}
  	\umin{(\parU,\parD)\in (\RR_+)^{\taillegrid}\times(\RR_+)^{\taillegrid}}  
    \la \normb{\vecrl}_1 + \frac{1}{2}\normH{y-\Cop\vecrl}^2  \tag{$\tilde{\Qq}_\la(y)$}\label{eq-reparam-cbpasso}
\end{align}
\begin{align}
\qandq
  \umin{(\parU,\parD)\in (\RR_+)^{\taillegrid}\times (\RR_+)^{\taillegrid}} 
  	\normb{\vecrl}_1 
	\quad\mbox{such that}\quad 
  \Cop\vecrl=y_0,  \tag{$\tilde{\Qq}_0(y_0)$}\label{eq-reparam-cbp}
\end{align}
where $\Cop \eqdef \begin{pmatrix}\OpU+\frac{h}{2} \OpD & \OpU-\frac{h}{2}\OpD\end{pmatrix}:\RR^{2\taillegrid}\rightarrow \Hh$.

The ``support recovery'' of $(\veccontO,\shiftcontO)$ through~\eqref{eq-abstract-cbpasso} is equivalent to the support recovery of $(\parU_0,\parD_0)$ through~\eqref{eq-reparam-cbpasso}.
As described below, the characterization of minimizers and the support recovery properties of the \lasso with positivity constraint~\eqref{eq-reparam-cbpasso} are quite similar to those of the classical \lasso described in the companion paper~\cite[Section 2]{2016-duval-thinlasso}. 
 
Since the regularization term is of the form 
\begin{align*}
  \quad J(\parU,\parD) &\eqdef 
  %\left\{\begin{array}{cl}
 % 	\sum_{i=0}^{\taillegrid-1} (\parU_i+\parD_i) &\mbox{ if } \parU_i\geq 0 \mbox{ and } \parD_i\geq 0 \mbox{ for all } i\in\seg{0}{\taillegrid-1},\\
 %   +\infty     & \mbox{ otherwise.}\end{array} \right.\\
  \sum_{i=0}^{\taillegrid-1} j(\parU_i) + \sum_{i=0}^{\taillegrid-1}j(\parD_i),   \qwithq j(x) \eqdef \sup 
  	\enscond{ qx }{q\leq 1} = 
	\left\{
		\begin{array}{cc}x & \mbox{if } x\geq 0,\\+\infty &\mbox{otherwise,}\end{array}
	\right.
\end{align*}
its subdifferential is the product of the subdifferentials $\partial j(\parU_i)$ and $\partial j(\parD_i)$ for $1\leq i\leq \taillegrid-1$, where
\begin{align*}
\partial j(x)&=\left\{\begin{array}{cc}\{1\} & \mbox{if } x> 0,\\ (-\infty, 1] &\mbox{if } x=0.\end{array}\right.
\end{align*}
That is quite similar to the subdifferential of $|\cdot|$ at $x\in \RR$ which is $-1$, $[-1,1]$ or $1$ if $x<0$, $x=0$ or $x>0$ respectively. Hence, one may adapt the standard results for the \lasso to the \lasso with positivity constraint, simply by replacing the conditions $\normi{\eta}\leq 1$ with $\max \eta \leq 1$ (and similarly for strict inequalities) wherever they appear. We leave the detail to the reader, and in the following, we use those results freely to derive the properties of the \lasso with cone constraint~\eqref{eq-abstract-cbpasso}.


%%%%%%%%%%%%%%%%%%%%%%%%%%%%%%%%%%%%%%%%%%%%%%%%%%%%%%
\subsection{Optimality conditions}
The optimality conditions for~\eqref{eq-reparam-cbpasso} and~\eqref{eq-reparam-cbp}, written in terms of $(\veccont_\la,\shiftcont_\la)$, yield the following results.

\begin{prop}
  Let $y\in \Hh$, $(\veccont_\la,\shiftcont_\la) \in \coneh^\taillegrid$, and $I=I(\veccont_\la,\shiftcont_\la)$. Then $(\veccont_\la,\shiftcont_\la)$ is a solution to~\eqref{eq-abstract-cbpasso} if and only if there exists $p_\la\in \Hh$ such that
\begin{align}
  \max\left( (\OpU^*+\frac{h}{2}\OpD^*)p_\la\right) \leq 1, \qandq \max\left( (\OpU^*-\frac{h}{2}\OpD^*)p_{\la}\right) \leq 1,\\
  (\OpU_{\Iup}^*+\frac{h}{2}\OpD_{\Iup}^*)p_{\la} =\bun_{\Iup}, \qandq (\OpU_{\Idown}^*-\frac{h}{2}\OpD_{\Idown}^*)p_{\la} = \bun_{\Idown},\label{eq-optimal-cbp-subdiff}\\
  \la \begin{pmatrix}\OpU^* \\ \OpD^* \end{pmatrix}p_{\la} + \begin{pmatrix}\OpU^* \\ \OpD^*\end{pmatrix}(\OpU \veccont_\la +\OpD \shiftcont_\la-y) =0. \label{eq-optimal-cbp-lagrange}
\end{align}
Similarly,  $(\veccontO,\shiftcontO)\in \coneh^\taillegrid$ is a solution to~\eqref{eq-abstract-cbp} if and only if $\OpU\veccontO + \OpD\shiftcontO=y_0$ and there exists $p\in \Hh$ such that
 \begin{align}
   \max\left( (\OpU^*+\frac{h}{2}\OpD^*)p\right) \leq 1, \qandq \max\left((\OpU^*-\frac{h}{2}\OpD^*)p\right) \leq 1,\\
  (\OpU_{\Iup}^*+\frac{h}{2}\OpD_{\Iup}^*)p =\bun_{\Iup}, \qandq (\OpU_{\Idown}^*-\frac{h}{2}\OpD_{\Idown}^*)p \leq \bun_{\Idown},
\end{align}
  where $I=I(\veccontO,\shiftcontO)$.
 \label{prop-optim-cbp}
\end{prop}

If the inequalities outside the support are strict, it is possible to ensure the uniqueness of the solution.

\begin{prop}
  Under the hypotheses of Proposition~\ref{prop-optim-cbp}, if $\begin{pmatrix}(\OpU+\frac{\stepsize}{2}\OpD)_{\Iup} &(\OpU-\frac{\stepsize}{2}\OpD)_{\Idown}\end{pmatrix}$ has full rank and if $p_\la$ (resp. $p$) satisfies
  \begin{align}
\forall k\in I^c, \quad  (\OpU^*p_\la)_k+ \frac{\stepsize}{2}|(\OpD^*p_\la)_k| <1,\\
\forall i\in \Idown\setminus\Iup,\quad  ((\OpU^* + \frac{\stepsize}{2}\OpD^*)p_\la)_i <1,\\
\forall i\in \Iup\setminus\Idown,\quad  ((\OpU^*- \frac{\stepsize}{2}\OpD^*)p_\la)_i <1,
   \end{align}
   then $(\veccont_\la,\shiftcont_\la)$ (resp. $(\veccontO,\shiftcontO)$) is the unique solution to~\eqref{eq-abstract-cbpasso} (resp.~\eqref{eq-abstract-cbp}).
\label{prop-cbp-strict}
\end{prop}

One may interpret the optimality conditions of Proposition~\ref{prop-optim-cbp} as the primal-dual relations between the solutions of~\eqref{eq-abstract-cbpasso} and~\eqref{eq-abstract-cbp} with the solutions of their respective dual problems
\begin{align}
  \inf_{p\in D} &\normH{\frac{y}{\la}-p}^2 \tag{$\Ee_\la(y)$}\label{eq-abstract-dual-cbpasso}\\
  \sup_{p\in D} &\langle y, p\rangle \tag{$\Ee_0(y)$}\label{eq-abstract-dual-cbp}\\
  \text{where } D&\eqdef\enscond{p\in \Hh}{\!\!\!\!\max_{k\in\seg{0}{\taillegrid-1}}(\OpU^*p)_k+\frac{\stepsize}{2}|(\OpD^*p)_k| \leq 1}.
\end{align}
Conversely, $p\in\Hh$ is a solution to \eqref{eq-abstract-dual-cbpasso} (resp.~\eqref{eq-abstract-dual-cbp}) if and only if $p\in D$ and $p$ satisfies the conditions of Proposition~\ref{prop-optim-cbp}.
%Conversely, if $(\parUO,\parDO)$) is the unique solution to~\eqref{eq-reparam-cbp}, then $\begin{pmatrix}\CopU_{\Iup} &\CopD_{\Idown}\end{pmatrix}$ has full rank and there exists $p\in \Hh$ such that~\eqref{eq-cbp-strict} holds.
%\begin{proof}
%  It suffices to prove the result for~\eqref{eq-abstract-cbp} since every solution $u_\la$ of~\eqref{eq-abstract-cbpasso} yields the same value of $\OpT u_\la$ and $u_\la$ is a solution to $\Qq_0(\OpT u_\la)$ (with the certificate $\OpT^*p_\la$).

 % Let $\hat u$, $u^\ast$ be any solution of \eqref{eq-abstract-cbp}, and let $(\hat\certcont,\hat \certshift)$ be a certificate for $\hat u$ such that~\eqref{eq-cbp-strict} holds with $I^c(\hat u)$. From the extremality conditions for problems in duality (see~\cite{ekeland1976convex}), we see that $(\hat\certcont,\hat \certshift)$ is also a dual certificate for $u^\ast$.
%  Hence using $\langle \hat\certcont, \certcont^\ast\rangle + \langle\hat \certshift, \tau^\ast\rangle = \sum_{j=0}^{\taillegrid-1}\certcont^\ast_j$, we obtain that $\certcont^\ast_j=0$ for all $j\in I^c(\hat u)$. 
%We conclude using the injectivity assumption.
%\end{proof}

%%%%%%%%%%%%%%%%%%%%%%%%%%%%%%%%%%%%%%%%%%%%%%%%%%%%%%
\subsection{Low noise behavior of C-BP}

The theorem of Fuchs~\cite{fuchs2004on-sp} for the \lasso describes an almost necessary and sufficient condition for the support stability of the problem at low noise. Its adaptation to the positive \lasso is straightforward. 
%\begin{prop}\label{prop-fuchs-cbp}
%  Let $(\veccontO,\shiftcontO)\in \coneh^\taillegrid\setminus\{0\}$ such that 
%  \eq{
%  	\Coph\eqdef\begin{pmatrix}(\OpU+\frac{\stepsize}{2}\OpD)_{\Iup} &(\OpU-\frac{\stepsize}{2}\OpD)_{\Idown}\end{pmatrix}
%	} 
%	has full rank, and let 
%\begin{align}
%	T& \eqdef \min \enscond{\shiftcontOi+\frac{2}{\stepsize}\veccontOi}{i\in \Iup(\shiftcontO,\veccontO)}\cup \enscond{\shiftcontOi-\frac{2}{\stepsize}\veccontOi }{i\in\Idown(\shiftcontO,\veccontO)}.\end{align}
%Then there exists constants $C^{(1)}>0,C^{(2)}>0$ such that for $\la \leq C^{(1)} T$, $\|w\|< C^{(2)}\la$, the solution to~$(\Qq_\la(y_0+w))$ is unique, satisfies $\Iup(\shiftcont_\la,\veccont_\la)=
%\Iup(\shiftcontO,\veccontO)$, $\Idown(\shiftcont_\la,\veccont_\la)=\Idown(\shiftcontO,\veccontO)$, and it reads:
%\begin{align*}
%  \begin{pmatrix}
%  \veccont_\la \\ \shiftcont_\la
%\end{pmatrix} = \begin{pmatrix}
%  \veccontO \\ \shiftcontO
%\end{pmatrix} +H_\stepsize\Coph^+ w - \la H_\stepsize(\Coph^*\Coph)^{-1}s,
%%\mbox{where } G=\begin{pmatrix}(\OpU+\frac{\stepsize}{2}\OpD)_{\Iup} &(\OpU-\frac{\stepsize}{2}\OpD)_{\Idown}\end{pmatrix}.
%\end{align*}
%where $H_\stepsize$ is defined in~\eqref{eq-def-H_h}, and $s\eqdef\bun_{\Iup\cup\Idown}$.
%\end{prop}
However, the criterion that it provides is not satisfied in general, and the support at low noise is strictly larger than $(\Iup(\veccontO,\shiftcontO),\Idown(\veccontO,\shiftcontO))$. We provide below a finer description of that support by studying the minimal norm certificate.

\begin{defn}[Minimal norm certificate]
  Let $(\veccontO,\shiftcontO)\in \coneh^\taillegrid$. Its minimal norm certificate is $\veccertcontO \eqdef \begin{pmatrix}
    \OpU^* +\frac{\stepsize}{2} \OpD^*\\
    \OpU^* -\frac{\stepsize}{2} \OpD^*
  \end{pmatrix} p_0$ where $p_0\in\Hh$ is the solution to~\eqref{eq-abstract-dual-cbp} with minimal norm.
  The extended support is $\extud (\veccontO,\shiftcontO)=\left(\extu(\veccontO,\shiftcontO),\extd(\veccontO,\shiftcontO)\right)$, where 
\begin{align}
  \extu(\veccontO,\shiftcontO)&=\enscond{j\in\seg{0}{\taillegrid-1}}{((\OpU^* +\frac{\stepsize}{2} \OpD^*)p_0)_j=1  },\\
  \extd(\veccontO,\shiftcontO)&=\enscond{j\in\seg{0}{\taillegrid-1}}{ ((\OpU^* -\frac{\stepsize}{2} \OpD^*)p_0)_j=1}.
\end{align}
\end{defn}

From the optimality conditions, if $(\veccontO,\shiftcontO)$ is a solution of~\eqref{eq-abstract-cbp} then $\Iup \subset \extu(\veccontO,\shiftcontO)$ and $\Idown\subset \extd(\veccontO,\shiftcontO)$
(where $I=I(\veccontO,\shiftcontO)$), and $p_0$ can be characterized as
\begin{align}
    p_0 &= \uargmin{q\in \Hh} \enscond{\|q\|}{\begin{pmatrix}
    \OpU^* +\frac{\stepsize}{2} \OpD^*\\
    \OpU^* -\frac{\stepsize}{2} \OpD^*
  \end{pmatrix} q\in\partial J(\parUO,\parDO)}.
\end{align}

We are now in position to describe the behavior of~\eqref{eq-abstract-cbpasso} at low  noise in the generic case. Some details about the proofs of the following Theorem can be found in~\ref{sec-continuous-abstract-proofs}. 

\begin{thm}\label{thm-abstract-cbp}
  Let $(\veccontO,\shiftcontO)\in \coneh^\taillegrid\setminus\{0\}$ be an identifiable signal, 
  	$(\Jup,\Jdown) \eqdef \extud(\veccontO,\shiftcontO)$ such that
   $\Copt\eqdef\begin{pmatrix}(\OpU+\frac{\stepsize}{2}\OpD)_{\Jup} &(\OpU-\frac{\stepsize}{2}\OpD)_{\Jdown}\end{pmatrix}$ has full rank.
   %
   Let  
   \eq{
   		\begin{pmatrix} u_{\Jup}\\v_{\Jdown} \end{pmatrix}\eqdef-(\Copt^*\Copt)^{-1}s
		\qwhereq
		s \eqdef (1, \ldots, 1)^* \in \RR^{|\Jup|+|\Jdown|},
  	}
  and assume that the following non-degeneracy condition holds
  \eql{\label{eq-non-degen}
  	\foralls j\in \Jup\setminus \Iup, \; u_j>0, 
	\qandq
	\foralls j\in \Jdown\setminus \Idown, \; v_j> 0.
  }
  Then, there exists constants $C^{(1)}>0$, $C^{(2)}>0$ such that for 
\begin{align}
\la &\leq C^{(1)} \min\left(\enscond{\shiftcontOi+\frac{2}{\stepsize}\veccontOi}{i\in \Iup(\veccontO,\shiftcontO)}\cup \enscond{\shiftcontOi-\frac{2}{\stepsize}\veccontOi }{i\in\Idown(\veccontO,\shiftcontO)}\right)\end{align}
and $\normH{w}\leq C^{(2)}\la$, the solution $(\veccont_\la,\shiftcont_\la)$ to~\eqref{eq-abstract-cbpasso} is unique, $\Iup(\veccont_\la,\shiftcont_\la)=\Jup$, $\Idown(\veccont_\la,\shiftcont_\la)=\Jdown$, and it reads
\begin{align*}
  \begin{pmatrix}
  \veccont_\la \\ \shiftcont_\la
\end{pmatrix} = \begin{pmatrix}
  \veccontO \\ \shiftcontO
\end{pmatrix} +H_\stepsize\Copt^+ w - \la H_\stepsize(\Copt^*\Copt)^{-1}s,
\end{align*}
where $H_\stepsize$ is defined in~\eqref{eq-def-H_h}.
\end{thm}



